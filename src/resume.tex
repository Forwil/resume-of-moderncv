%# -*- coding:utf-8 -*-
%% start of file `template_en.tex'.
%% Copyright 2006-1008 Xavier Danaux (xdanaux@gmail.com).
%
% This work may be distributed and/or modified under the
% conditions of the LaTeX Project Public License version 1.3c,
% available at http://www.latex-project.org/lppl/.


\documentclass[11pt,a4paper]{moderncv}
\usepackage{amssymb}
\usepackage{scalerel}
\usepackage{fontspec,xunicode}
\setmainfont{Tahoma}
\usepackage{xkeyval}
\usepackage[slantfont,boldfont]{xeCJK}
%\usepackage{xcolor}                 % replace by the encoding you are using
%\setCJKmainfont{Adobe Song Std L}  % 如果有安装Adobe Song 字体,就用这个吧
\setCJKmainfont{SimSun}             % 没有安装Adobe Song 字体的就换回SimSun字体吧
\setCJKfamilyfont{song}{SimSun}
\defaultfontfeatures{Mapping=tex-text}
\XeTeXlinebreaklocale "zh"
\XeTeXlinebreakskip = 0pt plus 1pt minus 0.1pt
% moderncv themes
\moderncvtheme[blue]{classic}                 % optional argument are 'blue' (default), 'orange', 'red', 'green', 'grey' and 'roman' (for roman fonts, instead of sans serif fonts)
%\moderncvtheme[green]{classic}                % idem

% character encoding


% adjust the page margins
\usepackage[scale=0.8]{geometry}
%\setlength{\hintscolumnwidth}{3cm}						% if you want to change the width of the column with the dates
%\AtBeginDocument{\setlength{\maketitlenamewidth}{6cm}}  % only for the classic theme, if you want to change the width of your name placeholder (to leave more space for your address details
\AtBeginDocument{\recomputelengths}                     % required when changes are made to page layout lengths

% personal data
\firstname{余}
\familyname{锋伟}
\title{个人简历}               % optional, remove the line if not wanted
\address{北京}{中国}    % optional, remove the line if not wanted
%\mobile{}                    % optional, remove the line if not wanted
\phone{18810 676 076}                      % optional, remove the line if not wanted
%\fax{fax (optional)}                          % optional, remove the line if not wanted
\email{forwil@foxmail.com}                     % optional, remove the line if not wanted
\extrainfo{github.com/forwil} % optional, remove the line if not wanted
%\photo[64pt]{picture}                         % '64pt' is the height the picture must be resized to and 'picture' is the name of the picture file; optional, remove the line if not wanted
%\quote{China\TeX 您的LaTeX乐园,TeX\&\LaTeX 王国}                 % optional, remove the line if not wante

%\nopagenumbers{}                             % uncomment to suppress automatic page numbering for CVs longer than one page


%----------------------------------------------------------------------------------
%            content
%----------------------------------------------------------------------------------
\begin{document}
\maketitle

\section{教育背景}
\cventry{2011.9--2012.7}{\emph{本科:}北京航空航天大学}{数学与系统科学学院}{华罗庚数学实验班}{}
{NOIp保送入学,大一结束后转系进入计算机学院。}
\cventry{2012.9--2015.7}{\emph{本科:}北京航空航天大学}{计算机学院}{计算机学院创新实验班}{}
{核心课程平均分:88/100。
%被选为毕业生代表与校长进行茶话座谈(共20人)。
本科综合排名7/228,获得研究生推免资格。}
\cventry{2015.9--至今}{\emph{研究生:}北京航空航天大学}{计算机学院}{软件工程}{}
{学位课程平均分:90.0/100,排名:6/241。}
\cvline{资格认证}{\emph{CCF:}计算机软件能力认证,成绩排名前 2.23\%}
\section{获奖经历}
\cventry{2009}{一等奖}{全国信息学奥林匹克联赛(NOIp)}{福建赛区}{325/400 第七名}{}
\cventry{2010}{一等奖}{全国信息学奥林匹克联赛(NOIp)}{福建赛区}{310/400}{}
\cventry{2011}{称号}{第十一届“福建省小科学家”}{福建}{}{}
%\cventry{2012}{二等奖}{北航第八届程序设计竞赛}{校级}{}{}
\cventry{2013}{二等奖}{高教社杯全国大学生数学建模竞赛}{全国}{}{}
\cventry{2014}{二等奖}{蓝桥杯全国软件大赛}{全国}{}{}
\cventry{2015}{一等奖}{ASC15 世界大学生超级计算机竞赛}{国际}{第五名}{}
\cventry{2015,2016}{一等奖}{硕士研究生学业奖学金}{校级}{}{}
\cventry{2016}{奖学金}{“华为”奖学金}{校级}{}{}
\section{实习与项目经历}
\cventry{2016.12--至今}{毕业设计:基于深度学习的中文文本纠错}{北京航空航天大学}{软件所}{}
{传统基于分词和规则错词表的中文纠错系统已经被用到实际的中文纠错系统中去,但是其依赖于错词表和查错规则,具有比较低的可扩展性。我们收集了超过10亿字的语料,设计了基于word-embedding、char level、2-stack-LSTM、dropout的语言模型,并使用双向模型、拼音相似性等策略来提升识别精度。系统在SIGHAN2013上取得了很好的错误检测精度(F1:0.69)。并且实验证明了,该系统在不断增加语料数据的时候,可以不断提升模型精度。 }
\cventry{2016.3--至今}{见习研究员}{SenseTime 商汤科技}{研究中心}{智能视频组(Mentor:闫俊杰)}
%{维护视频人脸检测跟踪识别系统SenseFace-GPU-SDK,把系统移植到NVIDIA Tegra X1平台上,优化后满足实时性要求,于2016北京安博会展出,并协助NVIDIA定制CES2017消费电子展demo。完成SenseFace-CPU版本的并行化方案设计、实现和优化,使系统最终可以实时处理多路1080P视频流。重写代码使得其满足公司自有的SDK规范,发布通用平台视频人脸检测跟踪识别系统SDK-video,提出一种基于跟踪的检测框再调整策略,可以大大降低检测次数,提升系统速度。把SDK-video移植到海思3519监控相机芯片中,做了大量优化,速度2hz提升至16hz,算法库交付给国内外多家监控厂商。协助实习生使用movidius芯片进行深度学习算法加速,完成caffe model到mvtensor的自动转换工具链,移植了人脸识别模型,帮助合作厂商宇视科技真正意义上把人脸检测跟踪识别算法做到了监控相机中去,极大降低了监控成本。实现并维护行人车辆视频结构化系统SenseObject-GPU-SDK,对性能做了大量优化。在MOT16(Multiple Object Tracking)上,使用行人检测和re-id特征优化了Tracking算法,取得包括MOTA(68.2和66.1)在内的多项指标世界第一,以第一作者身份发表一篇ECCV workshop paper。实习期间几乎每个月绩效分数为满分,实习9个月后拿到实习生全职offer。}  
{工程方面:编写、维护视频人脸检测跟踪识别系统SenseFace-GPU/CPU-SDK,视频结构化系统SenseVideo-GPU-SDK,并负责模型升级、框架并行、多线程/CUDA并行、INT8定点化加速、多平台移植和交付。作为监控算法嵌入式化的工程主要负责人,推动包括TX1/TX2(人脸跟踪识别服务器阵列)、海思3519(前端人脸抓拍相机)、Movidius芯片(前端人脸识别芯片)等前端产品落地。
算法方面:使用行人检测和ReID特征优化了多目标跟踪系统,在MOT16榜单上取得包括MOTA指标(68.2和66.1)在内的多项第一。
}
%\cventry{2015.9--2015.11}{项目}{多功能WiFi路由器}{课程设计}{https://github.com/Forwil/embedded-socks}{基于OpenWRT路由器固件和socks5代理协议,实现集齐信号增强、校园网免验证、自动翻墙三功能合一的无线路由器。实现简单web界面支持用户自行选择是否免验证、翻墙。同时提供android手机客户端。}
\cventry{2014.12--2015.5}{北航代表队队长}{ASC15 世界大学生超级计算机竞赛}{山西-太原}{}
{在初赛中:负责将4台浪潮服务器组成超算小集群的软硬件搭建和维护,对 HPCC 的多个测试子项目(包括 Linpack、FFT、DGEMM)进行深入分析和编译优化,撰写英文proposal,队伍以初赛大陆第一,世界第二进入全球总决赛。在总决赛中:负责集群软硬件平台搭建、功耗控制、HPL、HPCG调优、WRF-CHEM 应用优化和集群运行策略调度,最终队伍以全球第五名获得一等奖。} 
\cventry{2014.7--2014.12}{研发实习生}{Microsoft ARD 微软亚太研发集团,CEC - IoT Group}{北京}{}
{先后参与三个项目:1、在智能插座项目中,为 STM32F 上的.Net Micro Framework 固件添加高级 ADC 操作;2、在基于低功耗蓝牙的室内定位项目中,设计并实现在 51MCU 上的 RS-485 总线多对一通信协议;3、在自动化测试项目中,提取测试程序调用外部库的依赖关系,存入数据库并对外提供 WCF 接口。} 
%\cventry{2013.9--2013.12}{项目}{“向小葵”点评网}{课程设计}{https://github.com/Forwil/xxk}{使用 Python 的 WebPy 框架开发的点评书籍、电影和音乐的平台网站。前端采用 Bootstrap-UI 框架,数据库使用了 MySQL。负责前端界面设计与实现及后端数据处理和展示逻辑。队友设计了数据库触发器,存储过程。} 
%\cventry{2013.9--2014.1}{项目}{扩展PL/0文法的编译器}{课程设计}{https://github.com/Forwil/pl0ex}{使用标准 C 库,设计并实现了扩展 PL/0 文法的编译器(目标语言为 MIPS 汇编),包括全手写的词法分析、语法分析、语义分析,实现了中间四元式,寄存器分配,目标语言生成,公共子表达式优化。} 
%\cventry{2012.9--2013.1}{项目}{MIPS处理器设计}{课程设计}{https://github.com/Forwil/Mips-C}{使用 Verilog HDL,设计并实现了支持 55 条基本指令集的 MIPS 多周期处理器,指令包括基本四则运算、条件分支、函数调用与异常中断。} %}

\section{论文与专利}
%\nocite{*}
%\bibliographystyle{moderncv}
%\bibliography{bibtex}      % 'publications' is the name of a BibTeX file
\cvitem{2016}{\textbf{Fengwei Yu}, Wenbo Li, Quanquan Li, Yu Liu, Xiaohua Shi, and Junjie Yan. POI: Multiple Object Tracking with High Performance Detection and Appearance Feature[C]//European Conference on Computer Vision(\textbf{ECCV 2016}). Springer International Publishing, 2016: 36-42.}
\cvitem{2016}{专利:一种基于卷积神经网络特征的多目标在线跟踪算法(已受理)}
\cvitem{2016}{专利:基于目标特征点和产生式循环网络的多目标跟踪方法(已受理)}


\section{编程语言}
\cvlanguage{C/C++}{$\bigstar\bigstar\bigstar\bigstar$}{熟练使用其设计并实现高效算法,熟悉基本的编译/链接/运行过程}
\cvlanguage{Python}{$\bigstar\bigstar\bigstar$}{熟练使用其编写常用脚本/网页爬虫/网站后端}
%\cvlanguage{C\#/Java}{$\bigstar\bigstar$}{能够很好地运用面向对象编程范式编写可维护的软件}
%\cvlanguage{JavaScript}{$\bigstar\bigstar$}{能配合HTML/CSS,实现简单的网页功能,如交互、验证或配合Ajax实现高级操作,使用过HTML5相关特性(websocket,canvas)}
%\cvlanguage{Coq/Ocaml}{$\bigstar$}{理解函数式编程,了解如何编写可验证程序}

\section{相关技能}
\cvitem{Linux}{熟悉其软硬件环境安装,配置,编译链接,调试,习惯在Linux下工作}
\cvitem{Git}{能使用其对项目进行管理和维护,熟悉基本的远程仓库、分支功能}
\cvitem{算法与数据结构}{熟练掌握常见数据结构和算法,了解大多数高级数据结构,能正确估算程序的时间/空间复杂度}
\cvitem{并行计算调优}{理解并行编程基本方式,会使用MPI/OPENMP编写多进程/线程程序}
%\cvitem{网页前端/后端}{能够同时使用PHP/Python/JS/CSS/HTML/SQL编写完整的web应用}
\cvitem{深度学习}{熟悉基于CNN的物体检测、属性、识别等常见计算机视觉算法,熟悉各类深度学习模型inference框架,包括Caffe、cudnn、TensorRT等,知道如何在x86,arm,GPU,DSP上运行深度学习系统。}
%\cvitem{编程语言虚拟机}{了解Java虚拟机,阅读过Android上的ART虚拟机源码}
  
\section{学生工作经历}
\cventry{2013.11}{监考员}{全国信息学奥林匹克联赛(NOIp)}{北京赛区}{}
{监考普及组/提高组,负责解决考生遇到的编译/调试等问题}
\cventry{2013.9—2015.7}{班长}{北航计算机学院创新实验班}{}{}
{负责通知学生各类事宜,组织班会、聚餐等班级活动}
\cventry{2014.9—2015.1}{助教}{北航高等工程学院高等代数(1)}{本科课}{}
{负责批改作业、讲授习题课}
\cventry{2016.9—2017.1}{助教}{编译原理/形式语言与自动机}{本科课/研究生课}{}
{负责批改作业、小测验、实验课习题课讲解}
%% Publications from a BibTeX file

\end{document}

%% end of file `template_en.tex'.
