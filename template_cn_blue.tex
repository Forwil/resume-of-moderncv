%# -*- coding:utf-8 -*-
%% start of file `template_en.tex'.
%% Copyright 2006-1008 Xavier Danaux (xdanaux@gmail.com).
%
% This work may be distributed and/or modified under the
% conditions of the LaTeX Project Public License version 1.3c,
% available at http://www.latex-project.org/lppl/.


\documentclass[11pt,a4paper]{moderncv}
\usepackage{amssymb}
\usepackage{scalerel}
\usepackage{fontspec,xunicode}
\setmainfont{Tahoma}
\usepackage{xkeyval}
\usepackage[slantfont,boldfont]{xeCJK}
\usepackage{xcolor}                 % replace by the encoding you are using
%\setCJKmainfont{Adobe Song Std L}  % 如果有安装Adobe Song 字体,就用这个吧
\setCJKmainfont{SimSun}             % 没有安装Adobe Song 字体的就换回SimSun字体吧
\setCJKfamilyfont{song}{SimSun}
\defaultfontfeatures{Mapping=tex-text}
\XeTeXlinebreaklocale "zh"
\XeTeXlinebreakskip = 0pt plus 1pt minus 0.1pt
% moderncv themes
\moderncvtheme[blue]{classic}                 % optional argument are 'blue' (default), 'orange', 'red', 'green', 'grey' and 'roman' (for roman fonts, instead of sans serif fonts)
%\moderncvtheme[green]{classic}                % idem

% character encoding


% adjust the page margins
\usepackage[scale=0.8]{geometry}
%\setlength{\hintscolumnwidth}{3cm}						% if you want to change the width of the column with the dates
%\AtBeginDocument{\setlength{\maketitlenamewidth}{6cm}}  % only for the classic theme, if you want to change the width of your name placeholder (to leave more space for your address details
\AtBeginDocument{\recomputelengths}                     % required when changes are made to page layout lengths

% personal data
\firstname{余}
\familyname{锋伟}
\title{个人简历}               % optional, remove the line if not wanted
%\address{杭州}{中国}    % optional, remove the line if not wanted
%\mobile{}                    % optional, remove the line if not wanted
\phone{18810 676 076}                      % optional, remove the line if not wanted
%\fax{fax (optional)}                          % optional, remove the line if not wanted
\email{forwil@foxmail.com}                     % optional, remove the line if not wanted
\extrainfo{github.com/forwil} % optional, remove the line if not wanted
\photo[64pt]{picture}                         % '64pt' is the height the picture must be resized to and 'picture' is the name of the picture file; optional, remove the line if not wanted
%\quote{China\TeX 您的LaTeX乐园,TeX\&\LaTeX 王国}                 % optional, remove the line if not wante

%\nopagenumbers{}                             % uncomment to suppress automatic page numbering for CVs longer than one page


%----------------------------------------------------------------------------------
%            content
%----------------------------------------------------------------------------------
\begin{document}
\maketitle

\section{教育背景}
\cventry{2011.9--2012.7}{\emph{本科:}北京航空航天大学}{数学与系统工程学院}{华罗庚数学实验班}{}
{NOIp保送入学,大一结束转系进入计算机学院。}
\cventry{2012.9--2015.7}{\emph{本科:}北京航空航天大学}{计算机学院}{计算机学院创新实验班}{}
{校优秀毕业生,核心课程平均分:88/100。}
\cventry{2015.9--至今}{\emph{研究生:}北京航空航天大学}{计算机学院}{软件工程}{}
{本科综合排名7/228,保送入学。研究方向:形式化验证编译器、Java虚拟机}
\cvline{资格认证}{\emph{CCF:}计算机软件能力认证,成绩排名前 2.23\%}
\section{获奖经历}
\cventry{2009}{一等奖}{全国信息学奥林匹克联赛(NOIp)}{福建赛区}{325/400 第七名}{}
\cventry{2010}{一等奖}{全国信息学奥林匹克联赛(NOIp)}{福建赛区}{310/400}{}
\cventry{2011}{称号}{第十一届“福建省小科学家”}{福建}{}{}
\cventry{2012}{二等奖}{北航第八届程序设计竞赛}{校级}{}{}
\cventry{2013}{二等奖}{高教社杯全国大学生建模竞赛}{全国}{}{}
\cventry{2014}{二等奖}{蓝桥杯全国软件大赛}{全国}{}{}
\cventry{2015}{一等奖}{ASC15 世界大学生超级计算机竞赛}{国际}{第五名}{}
\section{实习与项目经历}
\cventry{2016.3--至今}{见习研究员}{SenseTime 商汤科技}{北京}{}
{负责维护一个动态人脸检测跟踪系统,完成CPU版本的并行化方案设计、实现和优化,使系统可以实时处理1080P视频流。并把系统移植到NVIDIA Tegra X1 嵌入式平台上。}  
\cventry{2014.12--2015.5}{北航代表队队长}{ASC15 世界大学生超级计算机竞赛}{太原}{}
{在初赛中对 HPCC 的多个测试子项目(包括 Linpack、FFT、DGEMM)进行深入分析和编译优化,撰写英文proposal,队伍以初赛大陆第一,世界第二进入全球总决赛。在总决赛中负责集群软硬件平台搭建与功耗控制、HPL、HPCG benchmark 调优与 WRF-CHEM 应用优化,最终队伍以全球第五名获得一等奖。} 
\cventry{2014.7--2014.12}{研发实习生}{微软亚太研发及团,CEC-IoT Group}{北京}{}
{先后参与三个项目:1、为 STM32F 上的.Net Micro Framework 固件添加高级 ADC 操作;2、设计并实现在 51MCU 上的 RS-485 总线通信协议;3、提取测试程序调用外部库的依赖关系,存入数据库并对外提供 WCF 接口。} 
\cventry{2013.9--2013.12}{项目}{“向小葵”点评网}{合作}{https://github.com/Forwil/xxk}
{使用 Python 的 WebPy 框架开发的点评书籍、电影和音乐的平台网站。前端采用 Bootstrap-UI 框架,数据库使用了 MySQL。负责前端界面设计与实现及后端数据处理和展示逻辑。队友设计了数据库触发器,存储过程。} 
\cventry{2013.9--2014.1}{项目}{扩展PL/0文法的编译器}{课程设计}{https://github.com/Forwil/pl0ex}
{使用标准 C 库,设计并实现了扩展 PL/0 文法的编译器(目标语言为 MIPS 汇编),包括全手写的词法分析、语法分析、语义分析,中间四元式,寄存器分配,目标语言生成,公共子表达式优化。} 
\cventry{2012.9--2013.1}{项目}{MIPS处理器设计}{课程设计}{https://github.com/Forwil/Mips-C}
{使用 Verilog HDL,设计并实现了支持 55 条基本指令集的 MIPS 多周期处理器,指令包括基本四则运算、条件
分支、函数调用与异常中断。} 
\section{编程语言}
\cvlanguage{C/C++}{$\bigstar\bigstar\bigstar\bigstar$}{熟练使用其设计并实现高效算法,熟悉基本的编译/链接/运行过程}
\cvlanguage{Python}{$\bigstar\bigstar\bigstar$}{熟练使用其编写常用脚本/网页爬虫/网页后端}
\cvlanguage{C\#/Java}{$\bigstar\bigstar$}{能够很好地运用面向对象编程范式编写可维护的软件}
\cvlanguage{JavaScript}{$\bigstar\bigstar$}{能配合HTML/CSS,编写出简单的网页功能,如交互、验证或配合Ajax实现高级操作,使用过websocket}
\cvlanguage{Coq/Ocaml}{$\bigstar$}{理解函数式编程,了解如何编写可验证程序}

\section{相关技能}
\cvlanguage{Linux}{}{熟悉其软硬件环境安装,配置,编译链接,调试}
\cvlanguage{Git}{}{会使用其对项目进行管理和维护,会使用简单的远程仓库、分支功能}
\cvlanguage{算法与数据结构}{}{熟练掌握常见数据结构和算法,了解大多数高级数据结构}
\cvlanguage{并行计算调优}{}{理解并行编程基本方式,会使用MPI/OPENMP编写多进程/线程程序}
\cvlanguage{网页前端/后端}{}{能够使用PHP/Python/JS/CSS/HTML/SQL,编写简单的网页前端+后端+数据库应用}
\cvlanguage{编程语言虚拟机}{}{了解Java虚拟机,阅读过Android上的ART虚拟机源码}
  
\section{学生工作经历}
\cventry{2013.11}{监考员}{全国信息学奥林匹克联赛(NOIp)}{北京赛区}{}
{监考普及组/提高组,负责解决考生遇到的编译/调试等问题}
\cventry{2013.9—2015.7}{班长}{北航计算机学院创新实验班}{}{}
{负责通知学生各类事宜,组织班会、聚餐等班级活动}
\cventry{2014.9—2015.1}{助教}{北航高等工程学院高等代数(1)}{}{}
{负责批改作业、习题课}


%% Publications from a BibTeX file
%\nocite{*}
%\bibliographystyle{plain}
%\bibliography{publications}       % 'publications' is the name of a BibTeX file

\end{document}


%% end of file `template_en.tex'.
