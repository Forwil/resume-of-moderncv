%# -*- coding:utf-8 -*-
%% start of file `template_en.tex'.
%% Copyright 2006-1008 Xavier Danaux (xdanaux@gmail.com).
%
% This work may be distributed and/or modified under the
% conditions of the LaTeX Project Public License version 1.3c,
% available at http://www.latex-project.org/lppl/.


\documentclass[11pt,a4paper]{moderncv}

\usepackage{fontspec,xunicode}
\setmainfont{Tahoma}
\usepackage{xkeyval}
\usepackage[slantfont,boldfont]{xeCJK}
\usepackage{xcolor}                 % replace by the encoding you are using
%\setCJKmainfont{Adobe Song Std L}  % 如果有安装Adobe Song 字体,就用这个吧
\setCJKmainfont{SimSun}             % 没有安装Adobe Song 字体的就换回SimSun字体吧
\setCJKfamilyfont{song}{SimSun}
\defaultfontfeatures{Mapping=tex-text}
\XeTeXlinebreaklocale "zh"
\XeTeXlinebreakskip = 0pt plus 1pt minus 0.1pt
% moderncv themes
\moderncvtheme[blue]{classic}                 % optional argument are 'blue' (default), 'orange', 'red', 'green', 'grey' and 'roman' (for roman fonts, instead of sans serif fonts)
%\moderncvtheme[green]{classic}                % idem

% character encoding


% adjust the page margins
\usepackage[scale=0.8]{geometry}
%\setlength{\hintscolumnwidth}{3cm}						% if you want to change the width of the column with the dates
%\AtBeginDocument{\setlength{\maketitlenamewidth}{6cm}}  % only for the classic theme, if you want to change the width of your name placeholder (to leave more space for your address details
\AtBeginDocument{\recomputelengths}                     % required when changes are made to page layout lengths

% personal data
\firstname{China}
\familyname{\TeX}
\title{网站简历}               % optional, remove the line if not wanted
%\address{杭州}{中国}    % optional, remove the line if not wanted
\mobile{123456789}                    % optional, remove the line if not wanted
\phone{000-238217878}                      % optional, remove the line if not wanted
%\fax{fax (optional)}                          % optional, remove the line if not wanted
\email{ChinaTeXer@gmail.com}                     % optional, remove the line if not wanted
\extrainfo{www.chinatex.org} % optional, remove the line if not wanted
\photo[64pt]{picture}                         % '64pt' is the height the picture must be resized to and 'picture' is the name of the picture file; optional, remove the line if not wanted
\quote{China\TeX 您的LaTeX乐园,TeX\&\LaTeX 王国}                 % optional, remove the line if not wante

%\nopagenumbers{}                             % uncomment to suppress automatic page numbering for CVs longer than one page


%----------------------------------------------------------------------------------
%            content
%----------------------------------------------------------------------------------
\begin{document}
\maketitle

\section{基本信息}
\cvline{网站名称}{\emph{China\TeX}}
\cvline{建站时间}{2002年}
\cvline{网站意义}{我们努力打造全新的\TeX\&\LaTeX 交流平台,建设中国独特的\TeX\&\LaTeX 社区,并为TeXer用户服务。欢迎大家光临,并提供指导性意见和建议!}
\section{站名由来}
\cventry{2002-2004}{2002年}{薛老师指导网站工作}{网站定名为China\TeX}{谨遵薛老师之建议,本站正式改名为ChinaTeX,欢迎薛定宇教授前来本站指导工作}{备注}
\section{论坛历史}
\cvline{title}{\emph{Title}}
\cvline{supervisors}{Supervisors}
\cvline{description}{\small Short thesis abstract}

\section{名字来由}
\subsection{Vocational}
\cventry{year--year}{Job title}{Employer}{City}{}{Description}                % arguments 3 to 6 are optional
\cventry{year--year}{Job title}{Employer}{City}{}{Description}                % arguments 3 to 6 are optional
\subsection{Miscellaneous}
\cventry{year--year}{Job title}{Employer}{City}{}{Description line 1\newline{}Description line 2}% arguments 3 to 6 are optional

\section{Languages}
\cvlanguage{language 1}{Skill level}{Comment}
\cvlanguage{language 2}{Skill level}{Comment}
\cvlanguage{language 3}{Skill level}{Comment}

\section{Computer skills}
\cvcomputer{category 1}{XXX, YYY, ZZZ}{category 4}{XXX, YYY, ZZZ}
\cvcomputer{category 2}{XXX, YYY, ZZZ}{category 5}{XXX, YYY, ZZZ}
\cvcomputer{category 3}{XXX, YYY, ZZZ}{category 6}{XXX, YYY, ZZZ}

\section{Interests}
\cvline{篮球}{\small 体力与技巧}
\cvline{hobby 2}{\small Description}
\cvline{hobby 3}{\small Description}

\renewcommand{\listitemsymbol}{-} % change the symbol for lists

\section{Extra 1}
\cvlistitem{Item 1}
\cvlistitem{Item 2}
\cvlistitem[+]{Item 3}            % optional other symbol

\section{Extra 2}
\cvlistdoubleitem[\Neutral]{Item 1}{Item 4}
\cvlistdoubleitem[\Neutral]{Item 2}{Item 5}
\cvlistdoubleitem[\Neutral]{Item 3}{}

%% Publications from a BibTeX file
%\nocite{*}
%\bibliographystyle{plain}
%\bibliography{publications}       % 'publications' is the name of a BibTeX file

\begin{thebibliography}{99}
\bibitem{11} LaTeX入门与提高,高等教育出版社。
\end{thebibliography}

\end{document}


%% end of file `template_en.tex'.
